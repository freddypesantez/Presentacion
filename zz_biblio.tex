\begin{thebibliography}{99}

\bibitem{Liu1} Liu G.P, 2010. Predictive Controller Design of Networked Systems With Communication Delays and Data Loss 

%\bibitem{Liu2} Liu G.P,Mu J. x, Rees D., Chai S. C., 2005. Design and Stability analysis of networrked control systmes with random communications time delay using the modified MPC. 

%\bibitem{Liu5} Liu G.P,Mu J. x, Rees D., Chai S. C., 2005. Design and Stability Criteria of Networked Predictive Control Systems With Random Network Delay in the Feedback Channel.

%\bibitem{Pin} Pin G., Parsini T. Networked Predictive Control of Uncertain Constrained Nonlinear Systems: Recursive Feasibility and Input-to-State Stability Analysis

%\bibitem{Liu4} Lui G.P., Chai S.C., Mu J.X. Rees D, 2008. Networked predictive control of systems with random delay in signal transmission channels.

%\bibitem{Onat} Onat A., Naskali T., Parlakay E, Mutluer O, 2011. Control Over Imperfect Networks: Model-Based Predictive Networked Control Systems

%\bibitem{TCP} Welzl J., 2005. Network Congestion Control, John Wiley \& Sons, Ltd. 

%\bibitem{Kumar} Thoma M., Allgower F., Morari M., Networked Control Systems.

%\bibitem{Kouva} Kouvaritakis B., Cannon, M., Model Predictive Control.

%\bibitem{Xia} Predictive control of networked systems with random delay and data dropout

% Wikipedia: Conector Hembra RJ45 \url{https://es.wikipedia.org/wiki/RJ-45}
%\bibitem{pares} Tipos de cables de red, 2010. \url{http://redeselie.blogspot.cl/2010/05/cableado-de-una-red-principales-tipos_8613.html}
    
%\bibitem{conector} Conectar una Roseta RJ45 - Bricolaje del PC \url{http://www.pasarlascanutas.com/tester_cable_ethernet_rj45/tester_cable_ethernet_rj45_0002.htm}
    
%\bibitem{ponchea} Artefactos para Realizar Redes. \url{https://sanjuanboscoedu.files.wordpress.com/2011/03/artefactos-para-ponchar-un-cable-utp.pdf}
    
%\bibitem{colores} Código de Colores para Cables de Red con Conectores RJ45. \url{http://wiki.elhacker.net/redes/zona-fisica/codigo-de-colores-para-cables-de-red-con-conectores-rj45}
    
    
%Resumen general de IoT y todas las capas. Menciona problemas generales junto con una buena introducción. MUST BE
%\bibitem{hsuo12}
%    Hui Suo, Jiafu Wan, Caifeng Zou, Jianqi Liu,
%    \emph{Security in the Internet of Things: A review}, International Conference on Computer Science and Electronics Engineering (2012).

\end{thebibliography}

%\addtocontents{toc}{\protect\vspace*{\baselineskip}}



% \bibitem{lamport94}
%     Leslie Lamport, Homer Simpson,
%     \emph{\LaTeX: a document preparation system},
%     Addison Wesley, Massachusetts,
%     8, No. 4, 1617-1624 (2014)
% 
% \bibitem{tag}       %try to use AuthorYear, for example: Tsunekawa16
%     Author,
%     \emph{Title}
%     Editor,
%     Edition number,
%     Year

